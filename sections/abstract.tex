\begin{abstract}
Fast and efficient diagnosis of malaria cases is essential in efforts to eliminate the disease. The default to diagnose malaria is a microscopy exam. This process becomes problematic when cases happen in rural areas and experts cannot be present to make such a diagnosis. Automation of the diagnostical process with the use of an intelligent system that would recognize malaria parasites could aid in problem resolution. Several laboratories capture the images in low quality using a system of microscopes based on mobile devices. Due to the poor quality of this system, conventional algorithms do not process these images properly. The use of Deep Learning may be a way to improve the accuracy of the present method. nevertheless, such an approach usually requires a large number of training sets, which is the reason why is difficult to apply to diagnostic systems deep learning techniques, since the data are usually protected by medical confidentiality. The use of Generating Adverse Networks can help in generating data for training and improving the accuracy of deep learning models. This paper aims at synthesizing malaria blood smears image objects using  Deep Convolutional Generating Adversarial Networks in conjunction with deep learning models in order to improve accuracy in object detection of peripheral blood smears.  Six experiment where performed and  showed that with 2200 images generated by the DCGAN network the classifier obtained a accuracy of 100\% in 600 real images tested.
\keywords{Malaria  \and Generative Adversarial Network \and Deep Learning}
% \PACS{PACS code1 \and PACS code2 \and more}
% \subclass{MSC code1 \and MSC code2 \and more}
\end{abstract}