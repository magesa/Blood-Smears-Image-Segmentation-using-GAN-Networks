\section{Introduction}
\label{intro}


The global burden of malaria is enormous. In 2012, the World Health Organization (WHO) estimated at least 247 million of worldwide people suffer from malaria and more than two billion or 42\% of of worldwide people has a risk of malaria because of living in malaria endemic areas, 627,000 of which resulted in deaths among African children. In the Philippines, malaria is considered to be the 9th leading cause of morbidity, with 58 out of the 81 provinces being malaria-endemic. Among the major obstacles for malaria eradication are the remote location of the majority of malaria cases and the lack of trained individuals that can analyze blood samples using
microscopy. The gold standard test for malaria is the method of preparing a blood smear on a glass slide, staining it, and examining it under a microscope. While several rapid diagnostic tests are also currently available, they still have shortcomings compared to microscopical analysis \cite{Quinn2016DeepDiagnostics}\cite{Premaratne2006AFilms}\cite{Penas2017}.


This is where automated systems for diagnosis
come in. Instead of manually going over a blood sample and
checking for the presence of malaria parasites, photographs
of the sample viewed from the microscope are analyzed by
an intelligent system. With such systems, the remote location
of malaria cases becomes less of a problem if such systems
become publicly available as trained microscopists and doctors \cite{Premaratne2006AFilms}\cite{Penas2017}.

Malaria is a disease caused by protozoa parasite of the genus Plasmodium that infects erythrocytes of patients. The one of Plasmodium species that infect humans is Plasmodium falciparum. This species is the most virulent because in a short time can invades erythrocytes in large numbers. Moreover, it causes various complications in the body's organs, and even causes the death. Most of these deaths were caused by the Plasmodium falciparum that infects red blood cells of patients which is characterized by a variety of organ dysfunction \cite{Dong2017}.


Malaria rapid diagnostic tests (RDTs) are relatively simple to perform and provide results quickly for making treatment decisions. However, the accuracy and application of RDT
results depends on several factors such as quality of the RDT, storage, transport and end user performance. A cross sectional survey to explore factors that affect the performance and use of
RDTs was conducted in the primary care facilities in South Africa \cite{Moonasar2007}


In the last decade, deep learning  methods have shown successful outcomes in different applications, including signal processing, object recognition, natural language processing, etc. Their attraction comes from the automatic feature learning ability in a deep network architecture 

Our objective was to develop an automated tool recognition of intracellular malaria parasites in stained blood films.   