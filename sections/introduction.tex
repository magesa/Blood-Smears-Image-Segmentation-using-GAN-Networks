\section{Introduction}
\label{intro}


The global burden of malaria is enormous. In 2012, the World Health Organization (WHO) estimated at least 247 million of worldwide people suffer from malaria and more than two billion or 42\% of of worldwide people has a risk of malaria because of living in malaria endemic areas, 627,000 of which resulted in deaths among African children. In the Philippines, malaria is considered to be the 9th leading cause of morbidity, with 58 out of the 81 provinces being malaria-endemic. Malaria is a disease caused by protozoa parasite of the genus Plasmodium that infects erythrocytes of patients. The one of Plasmodium species that infect humans is Plasmodium falciparum. This species is the most virulent because in a short time can invades erythrocytes in large numbers. Moreover, it causes various complications in the body's organs, and even causes the death. Most of these deaths were caused by the Plasmodium falciparum that infects red blood cells of patients which is characterized by a variety of organ dysfunction \cite{Dong2017}. Microscopic analysis of the blood smear image plays a very important role in the characterization of erythrocytes in the screening of malaria parasites. The characteristics of erythrocyte alterations due to parasitic malaria infection. The microscopic features of the erythrocyte include morphology, intensity and texture \cite{ShuleendaDevi2016}.



Among the major obstacles for malaria eradication are the remote location of the majority of malaria cases and the lack of trained individuals that can analyze blood samples using
microscopy. The gold standard test for malaria is the method of preparing a blood smear on a glass slide, staining it, and examining it under a microscope. While several rapid diagnostic tests are also currently available, they still have shortcomings compared to microscopical analysis \cite{Quinn2016DeepDiagnostics}\cite{Premaratne2006AFilms}\cite{Penas2017}. Several laboratories capture the images of low quality blood smears using a low cost microscope system based on a mobile application. Due to the poor quality of this system compared to traditional light-emitting microscopes, conventional algorithms do not adequately process these images \cite{Sorgedrager2018}.  The microscopic analysis of the blood smear by specialist is very tedious process and depends on the skill of the specialist clinician. To avoid this problem, different techniques of image analysis are being explored for diagnosis. The image analysis approach uses different differentiate between infected and uninfected erythrocytes. The microscopic feature used for the analysis of the characteristic erythrocytes to detect infected erythrocyte includes morphology, intensity and texture \cite{ShuleendaDevi2016}. 

This is where automated systems for diagnosis
come in. Instead of manually going over a blood sample and
checking for the presence of malaria parasites, photographs
of the sample viewed from the microscope are analyzed by
an intelligent system. With such systems, the remote location
of malaria cases becomes less of a problem if such systems
become publicly available as trained microscopists and doctors \cite{Premaratne2006AFilms}\cite{Penas2017}.

 

In the last decade, deep learning  methods have shown successful outcomes in different applications, including signal processing, object recognition, natural language processing, etc. Deep learning can be seen as an extension of well-known multilayer neural network classifiers trained with back propagation, except that many other layers are used. There are also different types of layers that are used in typical sequences. Deep learning utilizes massive amounts of computational power, among different deep learning methods, convolutional neural network (CNN) based algorithms are more suitable in image related tasks, since images have highly correlated intensities in local regions and some local signals or statistics are invariant to location \cite{Yan2017Multi-InstanceRecognition}. The idea of CNNs is to apply smaller convolutional kernels (or filters) in combination with a deep network architecture to capture the discernible image resources as much as possible. Their attraction comes from the automatic feature learning ability in a deep network architecture. 

Deep learning is the latest trend in machine learning, which has boosted performance in many non-medical areas. Deep learning usually requires large training sets. This is the reason why medical applications have been among the latest applications to embrace deep learning, as images are particularly difficult to obtain by the need for trained experts and privacy issues \cite{Dong2017a}. Malaria diagnosis is an ideal application for deep learning because segmented red blood images can be used as input to a convolutional neural network. Deep learning does not require the design of hand-made resources, which is one of its greatest advantages \cite{Liang2017}.

Generative adversarial networks (GANs) are deep neural net architectures comprised of two nets, pitting one against the other. GANs learn to mimic a distribution of data,  creating new samples in similar  domain. One neural network, called the generator, generates new data instances, while the other, the discriminator, evaluates them for authenticity; i.e. the discriminator decides whether each instance of data it reviews belongs to the actual training dataset or not  \cite{Goodfellow2014}.  Radford proposes a new  model called DCGAN(Deep Convolutional GAN) that uses convolutional layers in a GAN \cite{Radford2015UnsupervisedNetworks}.


Our objective was to use DCGAN networks to generate images of malaria blood smear objects to improve object detection. We develop a process for object detection in malaria blood smears using DCGAN networks to train convolutional neural networks. We performed six experiments that showed that the images generated by the DCGAN network can be used in the training of a CNN network to detect objects improving classifier accuracy.