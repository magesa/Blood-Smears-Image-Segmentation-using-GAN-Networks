\section{Introduction}
\label{intro}


The global burden of malaria is enormous. In 2012, the World Health Organization (WHO) estimated 207,000,000 cases of malaria globally, 627,000 of which resulted in deaths among African children. In the Philippines, malaria is considered to be the 9th leading cause of morbidity, with 58 out of the 81 provinces being malaria-endemic. Among the major obstacles for malaria eradication are the remote location of the majority of malaria cases and the lack
of trained individuals that can analyze blood samples using
microscopy. This is where automated systems for diagnosis
come in. Instead of manually going over a blood sample and
checking for the presence of malaria parasites, photographs
of the sample viewed from the microscope are analyzed by
an intelligent system. With such systems, the remote location
of malaria cases becomes less of a problem if such systems
become publicly available as trained microscopists and doctors \cite{Premaratne2006AFilms}\cite{Penas2017}.




Malaria rapid diagnostic tests (RDTs) are relatively simple to perform and provide results quickly for making treatment decisions. However, the accuracy and application of RDT
results depends on several factors such as quality of the RDT, storage, transport and end user performance. A cross sectional survey to explore factors that affect the performance and use of
RDTs was conducted in the primary care facilities in South Africa \cite{Moonasar2007}

Our objective was to develop an automated tool recognition of intracellular malaria parasites in stained blood films.   