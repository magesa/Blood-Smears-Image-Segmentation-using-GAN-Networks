\section{Deep Learning for Malaria Diagnosis}

In this project, a preliminary systematic mapping of the state of the art was conducted based on the process described by Petersen et al. \cite{Petersen2015} according to which, there are five essential steps to be followed: (i) definition of research questions, (ii) conducting research on relevant primary studies, (iii) document screening, (iv) extract keywords of abstracts, and (v) data extraction and mapping. Considering that research questions should exemplify the objectives of the mapping study, the following question was elaborated: Which research from 2002 to 2018 applied the use of deep learning in the diagnosis of malaria using peripheral blood smears?

Initially searches were performed with keywords consisted of the combination of the word Malaria, machine learning, deep learning and synonyms. In the first step, all primary studies retrieved were evaluated in order to identify those relevant to answer the research questions. After reading the titles, abstracts and keywords, this initial set was reduced to 16 articles that contained studies related to the use of deep learning.

Rajpurkar et al. have developed a reinforcement learning agent (RL) that can predict the probability of an individual positive test for malaria by asking questions about your home. After the learning phase, the agent determines the next question in the survey and uses stopping criteria to forecast malaria probability based on your responses so far \cite{Rajpurkar2017}. Gueorguieva et al. applied the Faster R-CNN object detection model to identify cells and recognize their stages in clear field microscopy images of malaria-infected blood \cite{hung2017applying}. Poostchi et al. presents a systematic review with a set of approaches for malaria automatic diagnosis \cite{Poostchi2018}.


Zhang et al. presented a two-step approach to detecting infected and uninfected cells. The first step applies an object-detection structure with trained classifier to detect all red blood cells in a blood drop image. The second stage classifies each segmented region into an infected or uninfected cell, applying morphological characteristics \cite{Zhang2016}. Liang et al. use a convolutional neural network to discriminate between infected and uninfected cells in fine blood smears after the application of a conventional cell segmentation approach classification \cite{Liang2017}.

Other authors who have applied deep learning in cell segmentation are Dong et al. and Gopakumar et al. who used convolutional neural networks. Dong et al. used whole images of thin blood slides to compile a dataset of red blood cells infected with malaria and uninfected cells as labeled by a group of four pathologists. Three types of known convolutional neural networks were evaluated, including LeNet, AlexNet and GoogLeNet. The simulation results showed that all these deep convolution neural networks achieved classification accuracies of more than 95 \%, greater than the accuracy of about 92 \% achievable using the support vector machine (SVM) method. In addition, deep learning methods have the advantage of being able to automatically learn the characteristics of the incoming data, thus requiring the least amount of human expert inputs for automated malaria diagnosis \cite {Dong2017} \cite{Dong2017a}. Bibin et al. used deep belief networks, and recently Hung et al. presented an end-to-end structure using faster convolutional neural network.

Premaratne et al. used digital images of oil immersion views from microscopic slides captured though a capture card. They were preprocessed by segmentation and grayscale conversion to reduce their dimensionality and later fed into a feed forward backpropagation neural network (NN) for training \cite{Premaratne2006AFilms}.

Among the studies investigated, the use of Generative Adversarial Networks for the generation of new samples of peripheral blood smears was not presented. The present research intends to use Generating Adversarial Networks to generate new samples of peripheral blood smears with the objective of improving the accuracy in automatic diagnosis in cases of malaria.



